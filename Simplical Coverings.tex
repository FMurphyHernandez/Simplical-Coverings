\documentclass{amsart}

\usepackage{lipsum}
\usepackage{amsfonts}
\usepackage{graphicx}
\usepackage{amsmath}
\usepackage{amssymb}
\usepackage[english]{babel}
\usepackage{tikz-cd}
\usepackage{hyperref}
\usepackage{enumerate}

%%%%%%%%%%%%%%%%%%%%%%%%%%%%%%%%%%%%%%%%%%%%%
%Topology and its Applications
%Rocky Mountain Journal
%%%%%%%%%%%%%%%%%%%%%%%%%%%%%%%%%%%%%%%%%%%%%

\newtheorem{theorem}{Theorem}[section]
\newtheorem{lemma}{Lemma}[section]
\newtheorem{proposition}{Proposition}[section]
\newtheorem{corollary}{Corollary}[section]
\newtheorem{definition}{Definition}[section]
\newtheorem{observation}{Observation}[section]
\newtheorem{example}{Example}[section]

\newenvironment{AMS}{****}{****}

\begin{document}

\title{Geometric Realization of Covering Complexes}

\author{Frank Murphy-Hernandez\\Luis Alberto Mac\'ias-Barrales}
\address{Facultad de Ciencias, UNAM, Mexico City}
\email{murphy@ciencias.unam.mx}

\author{Luis Mac\'ias-Barrales}
\address{Instituto de Matem\'aticas, UNAM, Mexico City}
\email{}

\subjclass[2000]{Primary ****, ****; Secondary ****, ****}

\date{\today}

\keywords{Abstract Simplicial Complexes, Coverings, Fundamental Group}

\begin{abstract}
We prove that the geometric realization of a covering complex is a covering space. Also, this holds true for the universal covering complex.
\end{abstract}

\maketitle

%%%%%%%%%%%%%%%%%%%%%%%%%%%%%%%%%%%%%%%%%%%%%%%%%%%%%%%%%%%%%%%%%%%%
\section*{Introduction}

The theory of abstract simplicial complexes is a useful tool in the calculation of fundamental groups. This fact appears explicitly in the paper \cite{weil1960discrete} of A. Weil, .....
For any connected abstract simplicial complex $S$ is edge-path group $E(S)$ is naturally isomorphic to the fundamental group of its geometric realization $\pi_1(\vert S\vert)$. The edge-path group could be described explicitly by  generators and relations. As reference for these facts, see the book of I. Singer and J. Thorpe \cite{singer2015lecture}.

We base our definition of covering complex given by J. Rotman in \cite{rotman1973covering}, but there are other definitions covering complex as the one in \cite{abello1991complexity}. The definition of J. Rotman

%%%%%%%%%%%%%%%%%%%%%%%%%%%%%%%%%%%%%%%%%%%%%%%%%%%%%%%%%%%%%%%%%%%%
\section{Preliminaries}


We recall that, given a topological space $X$, a covering space on $X$ it's a continuous map $p\colon E\to X$, such that for every $x\in X$, there is an open neighborhood $U$ such that $p^{-1}(U)$ it is a disjoint union of open sets $U_{\lambda}$, $\lambda\in\Lambda$, and $p|_{U_{\lambda}}\colon U_{\lambda}\to U$ it's a homeomorphism. We recommend the book of P. May \cite{may1999concise} and the book of J. Rotman \cite{rotman2013introduction} as reference of covering spaces.

An abstract simplicial complex is a pair $(S,\mathcal{K})$ where $S$ is a set and $\mathcal{K}$ is a family of non-empty finite subsets of $S$ such that:
\begin{itemize}
\item $\bigcup\mathcal{K}=S$.
\item If $\sigma\subseteq \tau$ and $\tau\in \mathcal{K}$ then $\sigma\in\mathcal{K}$.
\end{itemize}
We call complexes to the abstract simplicial complexes.  If $\sigma\in \mathcal{K}$, then the dimension of $\sigma$ is $\vert\sigma\vert-1$, and we denote it by $dim(\sigma)$. The elements of $\mathcal{K}$ of dimension $n$ are called $n$-simplices, and we denote the set of $n$-simplices by $\mathcal{K}_n$. The $0$-simplices are called vertices. The dimension of $(S,\mathcal{K})$ is defined as the supremum of $dim(\sigma)$ where $\sigma$ ranges over $\mathcal{K}$, we denote it by $dim(S,\mathcal{K})$. This dimension may be infinite. We call the complex $(S,\mathcal{K})$  finite, if $S$ is finite. In particular, a finite complex has finite dimension. A complex $(S,\mathcal{K})$ is called discrete, if $\mathcal{K}=\{\{s\}\mid s\in S\}$.

An edge $e$ in $(S,\mathcal{K})$ is a pair of vertices  $(x,y)$ where $\{x,y\}\in\mathcal{K}$, $x$ is the origin of the edge $e$ and we denote it by $orig(e)$, and $y$ is the end of the edge $e$ and we denote by $end(e)$. A path $\alpha$ in $(S,\mathcal{K})$ is a finite sequence of edges $e_1,\dots,e_n$ such that $end(e_i)=orig(e_{i+1})$ with $i=1,\dots,n-1$. We define $orig(\alpha)=orig(e_1)$ and $end(\alpha)=end(e_n)$.

A morphism between complex $(S_1,\mathcal{K}_1)$ and $(S_2,\mathcal{K}_2)$ is a map $f\colon S_1\longrightarrow S_2$ such that $f(\sigma)\in \mathcal{K}_2$ for any $\sigma\in\mathcal{K}_2$. We call a morphism of complexes a simplicial map. We  denote by $\mathcal{C}$ to the category of complexes and simplicial maps.

As reference of complexes, we recommend \cite{singer2015lecture}  and \cite{spanier1989algebraic}.

We  denote by $\mathcal{T}$ to the category of topological spaces and continuous maps.

An abstract simplicial complex is a pair $(S,\mathcal{K})$ where $S$ is a set and $\mathcal{K}$ is a family of non-empty finite subsets of $S$ such that, if $\sigma\subseteq \tau$, $\sigma\neq\emptyset$ and $\tau\in \mathcal{K}$ then $\sigma\in\mathcal{K}$. A morphism between abstract simplicial complexes $f\colon(S_1,\mathcal{K}_1)\to(S_2,\mathcal{K}_2)$ it's a funtion $f\colon S_{1}\to S_{2}$ such that $f(\sigma)\in\mathcal{K}_{2}$ for any simplex $\sigma\in\mathcal{K}_{1}$. We will denote $\mathcal{C}$ to the category of abstract simplicial complexes.

The geometric realization of an abstract simplical complex $(S,\mathcal{K})$ is given by the following formula: first we give a total order to $S$. Then, for any simplex $\sigma=\{s_{0}<s_{1}<\ldots<s_{q}\}$ we define $|\sigma|=\Delta^{q}$, the standar topological $q$-simplex and we asociate to the vertex $s_{q}$ the $q$-th vertex of $\Delta^{q}$. If $\tau=\{s_{0}<\ldots<s_{q}\}$ is a simplex and $\sigma=\{s_{q_{1}}<\ldots<s_{q_{k}}\}\subseteq\tau$, we define $i_{\sigma}^{\tau}\colon|\sigma|\to|\tau|$ to be the affine function such that maps the $j$-th vertex of $|\sigma|$ to the $q_{j}$-th vertex of $|\tau|$. Thus we take $|S|$ as the colimit over this system. If $f\colon(S_{1},\mathcal{K}_{1})\to(S_{2},\mathcal{K}_{2})$ it's a morphism of abstract simplicial complexes, then we can define $|f|\colon|S_{1}|\to|S_{2}|$ as the colimit of the affine functions $|f|_{\sigma}|\colon|\sigma|\to|f(\sigma)|$. $|-|\colon\mathcal{C}\to\mathcal{T}$ it's a functor, were $\mathcal{T}$ is the category of topological spaces and continuous maps.


We shall recall that, given a topological space $X$, a covering space on $X$ it's a continuous map $p\colon E\to X$, such that for every $x\in X$, there is an open neighborhood $U$ such that $p^{-1}(U)$ it's a disjoint union of open sets $U_{\lambda}$, $\lambda\in\Lambda$, and $p|_{U_{\lambda}}\colon U_{\lambda}\to U$ it's a homeomorphism.

%%%%%%%%%%%%%%%%%%%%%%%%%%%%%%%%%%%%%%%%%%%%%%%%%%%%%%%%%%%%%%%%%%%%
\section{Geometric Covering Complexes}


\begin{definition}
Let $(S,\mathcal{K})$ be a complex. We say that $(S,\mathcal{K})$ is connected if for any pair of vertices $x,y$ of $(S,\mathcal{K})$ there is a path $\alpha$ such that $orig(\alpha)=x$ and $end(\alpha)=y$.
\end{definition}

The following definition is due J. Rotman in \cite{rotman1973covering}.

\begin{definition}
Let $(S,\mathcal{K})$ be a complex. A covering of $(S,\mathcal{K})$ is a pair $((T,\mathcal{L}),p)$ where $(T,\mathcal{L})$ is a complex and $p\colon T\longrightarrow S$ is a simplicial map such that:
\begin{itemize}
\item $(T,\mathcal{L})$ is a connected complex.
\item For every $\sigma\in\mathcal{K}$, $p^{-1}(\sigma)=\bigcup_{i\in I}\sigma_i$ where the family $\{\sigma_i\}_{i\in I}$ is a pairwise disjoint family of simplices of $\mathcal{L}$ such that $p|_{\sigma_i}\colon \sigma_i\longrightarrow \sigma$ is bijective.
\end{itemize}
The map $p$ is called projection and the simplices $\sigma_i$ are called sheets over $\sigma$.
\end{definition}

We observe that $p$ is surjective and $(S,\mathcal{K})$ is connected. For our geometric porpoises we need a stronger definition of covering complex.

\begin{definition}
Let $(S,\mathcal{K})$ be a complex. A geometric covering of $(S,\mathcal{K})$ is a covering $((T,\mathcal{L}),p)$ such that for any simplex $\sigma\in\mathcal{L}_n$, $p(\sigma)\in\mathcal{K}_n$. In other words, $p$ preserves the dimension of the simplices. We have that geometric coverings preserve the dimension of the complexes.
\end{definition}


\begin{proposition}
Let $(S,\mathcal{K})$ be a complex, and $((T,\mathcal{L}),p)$ a covering complex of $(S,\mathcal{K})$. If $\sigma\in\mathcal{K}$ is a maximal simplex, then $p^{-1}$ is the disjoint union of maximal simplices.
\end{proposition}

\begin{proof}
Let $\sigma\in\mathcal{K}$ be a maximal simplex with  $p^{-1}(\sigma)=\bigcup_{i\in I}\sigma_i$ where the family $\{\sigma_i\}_{i\in I}$ is a pairwise disjoint family of simplices of $\mathcal{K}$ such that $p|_{\sigma_i}\colon \sigma_i\longrightarrow \sigma$ is bijective. If $\sigma_j$ is not maximal for some $j\in I$, then there is a simplex $\tau\in\mathcal{L}$ such that $\sigma_j\subset \tau$. As $\sigma$ is maximal and $\sigma=p(\sigma_j)\subseteq p(\tau)$, we have that $p(\tau)=\sigma$. This contradicts the fact that $p$ preserves the dimension. Therefore $\sigma_i$ is a maximal simplex.
\end{proof}

\begin{definition}
Let  $(S,\mathcal{K})$ and $(T,\mathcal{L})$ be complexes and $f\colon (S,\mathcal{K})\longrightarrow (T,\mathcal{L})$ be a simplicial map. We say that $f$ reflects maximal simplices, if $\sigma\in\mathcal{K}$ is such that $f(\sigma)$ is a maximal simplex, then $\sigma$ is a maximal simplex.
\end{definition}


\begin{proposition}
Let $(S,\mathcal{K})$ be a finite dimensional complex, and $((T,\mathcal{L}),p)$ a covering complex of $(S,\mathcal{K})$. Then $((T,\mathcal{L}),p)$ is a geometric covering complex of $(S,\mathcal{K})$ if and only if $p$ reflects maximal simplices.
\end{proposition}

\begin{proof}
Conjetura
\end{proof}




\begin{definition}
Let $(S,\mathcal{K})$ an abstract simplicial complex. An abstract simplicial covering on $(S,\mathcal{K})$ it's a simplicial map $p\colon(E,\mathcal{L})\to(S,\mathcal{K})$ such that for each simplex $\sigma\in\mathcal{K}$, $p^{-1}(\sigma)=\amalg_{\lambda\in\Lambda}\sigma_{\lambda}$ were $\sigma_{\lambda}\in\mathcal{L}$ and $p|_{\sigma_{\lambda}}\colon \sigma_{\lambda}\to \sigma$ it is onto and one to one, for each $\lambda\in\Lambda$.
\end{definition}

\begin{proposition}
Let $p\colon(E,\mathcal{L})\to(S,\mathcal{K})$ be an abstract simplicial covering. Then $|p|:|E|\to|S|$ it's a covering space.
\end{proposition}

%%%%%%%%%%%%%%%%%%%%%%%%%%%%%%%%%%%%%%%%%%%%%%%%%%%%%%%%%%%%%%%%%%%%
\section{Geometric Realization of Covering Complexes}

\begin{definition}[Geometric Realization]
The geometric realization of a complex $(S,\mathcal{K})$ is the set of all function $\phi\colon S\longrightarrow [0,1]$ such that:
\begin{itemize}
\item $supp(\phi)\in\mathcal{K}$
\item $\sum_{s\in S}\phi(s)=1$
\end{itemize}
We denote this set by $\vert (S,\mathcal{K})\vert$. We can give $\vert (S,\mathcal{K})\vert$ a metric topology given by:
\[
d(\phi,\psi)=\sqrt{\sum_{s\in S}(\phi(s)-\psi(s))^{2}}
\]
for $\phi,\psi\in \vert (S,\mathcal{K})\vert$. When we endowed $\vert (S,\mathcal{K})\vert$ with the metric topology we denote it by $\vert (S,\mathcal{K})\vert_d$. There is a second topology for $\vert (S,\mathcal{K})\vert$ called the coherent topology. Foer each simplex $\sigma\in\mathcal{K}$, we define its geometric realization $\vert\sigma\vert$ as the set of functions $\phi\in \vert (S,\mathcal{K})\vert$ with $supp(\phi)\subseteq \sigma$. We give to $\vert \sigma\vert$ the subspace topology inherited as subset of $\vert (S,\mathcal{K})\vert$. If we consider the inclusion $i_\sigma\colon\vert\sigma\vert\longrightarrow \vert (S,\mathcal{K})\vert$, then coherent topology on $\vert (S,\mathcal{K})\vert$ is the largest topology which makes all the inclusions continuous. Usually, $\vert (S,\mathcal{K})\vert$ is considered with the coherent topology. We may characterize $\vert (S,\mathcal{K})\vert$ as the colimit  in $\mathcal{T}$ of the geometric realization of its simplices. So a function $f\colon\vert (S,\mathcal{K})\vert\longrightarrow X$ is continuous if and only its restrictions to $\vert\sigma\vert$ is continuous for all $\sigma\in\mathcal{K}$. Especially, the identity $\vert (S,\mathcal{K})\vert\longrightarrow \vert (S,\mathcal{K})\vert_d$ is continuous, so the coherent topology contains the metric topology. In particular, if $(S,\mathcal{K})$ is a finite complex, then both topologies coincide.
\end{definition}


\begin{definition}
Let $X$ be a simplicial set. A simplicial covering on $X$ is a pair $(Y,p)$ where $Y$ is a simplicial set and $p\colon Y\longrightarrow X$ is a simplicial map such that
\end{definition}

\begin{definition}
If $f\colon (S,\mathcal{K})\longrightarrow (T,\mathcal{L})$ is a simplicial map, then it induces a continuous function $\vert f\vert\colon \vert(S,\mathcal{K})\vert\longrightarrow \vert(T,\mathcal{L})\vert$. If $\phi\in \vert(S,\mathcal{K})\vert$ with $\sigma=supp(\phi)$ then $\phi=\sum_{s\in\sigma}\phi(s)\phi_s$. So  we define $\vert f\vert (\phi):=\sum_{s\in \sigma}\phi(s)\phi_{f(s)}$. In this way, the geometric realization is a functor from $\mathcal{C}$ to $\mathcal{T}$.
\end{definition}

\begin{proposition}
Let $(S,\mathcal{K})$ be a complex and $((T,\mathcal{L}),p)$ a finite geometric covering of $(S,\mathcal{K})$. Then $\vert (T,\mathcal{L}), p)\vert $ is covering of $\vert (S,\mathcal{K})\vert$.
\end{proposition}

\begin{proof}
Let $\phi\in \vert (S,\mathcal{K})\vert$ with $supp(\phi)=\sigma\in\mathcal{K}$. Then $p^{-1}(\sigma)=\bigcup_{i\in I}\sigma_i$ where the family $\{\sigma_i\}_{i\in I}$ is a pairwise disjoint family of simplices of $\mathcal{L}$ such that $p|_{\sigma_i}\colon \sigma_i\longrightarrow \sigma$ is bijective.  We define $R=\min\{\frac{d(\phi,\phi_s}{2}\mid s\in\sigma\}$. So $\mathbb{B}_R(\phi)$ is an open neighborhood of $\phi$. We affirm that $\vert p\vert^{-1}(\mathbb{B}_R(\phi))=\bigcup_{i\in I}\mathbb{B}_R(\phi_i)$
\end{proof}

\begin{proposition}
Let $(S,\mathcal{K})$ be a complex and $((T,\mathcal{L}),p)$ a finite dimensional geometric covering of $(S,\mathcal{K})$. Then $\vert (T,\mathcal{L}), p)\vert $ is covering of $\vert (S,\mathcal{K})\vert$.
\end{proposition}

\begin{proof}

\end{proof}

\begin{proposition}
Let $(S,\mathcal{K})$ be a complex and $((T,\mathcal{L}),p)$ a geometric covering of $(S,\mathcal{K})$. Then $\vert (T,\mathcal{L}), p)\vert $ is covering of $\vert (S,\mathcal{K})\vert$.
\end{proposition}

\begin{proof}

\end{proof}


%%%%%%%%%%%%%%%%%%%%%%%%%%%%%%%%%%%%%%%%%%%%%%%%%%%%%%%%%%%%%%%%%%%%
\section{Geometric Realization of the Universal Covering Complex}



%%%%%%%%%%%%%%%%%%%%%%%%%%%%%%%%%%%%%%%%%%%%%%%%%%%%%%%%%%
\bibliography{biblio}
\bibliographystyle{plain}



\end{document}

%------------------------------------------------------------------------------
% End of journal.tex
%------------------------------------------------------------------------------
