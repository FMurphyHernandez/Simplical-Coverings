\documentclass{amsart}

\usepackage{lipsum}
\usepackage{amsfonts}
\usepackage{graphicx}
\usepackage{amsmath}
\usepackage{amssymb}
\usepackage[english]{babel}
\usepackage{tikz-cd}
\usepackage{hyperref}
\usepackage{enumerate}


\newtheorem{theorem}{Theorem}[section]
\newtheorem{lemma}{Lemma}[section]
\newtheorem{proposition}{Proposition}[section]
\newtheorem{corollary}{Corollary}[section]
\newtheorem{definition}{Definition}[section]
\newtheorem{observation}{Observation}[section]
\newtheorem{example}{Example}[section]

\newenvironment{AMS}{****}{****}

\begin{document}

\title{Simplicial Coverings}

\author{Frank Murphy-Hernandez}
\address{Facultad de Ciencias, UNAM, Mexico City}
\email{murphy@ciencias.unam.mx}

\subjclass[2000]{Primary ****, ****; Secondary ****, ****}

\date{\today}

\keywords{Simplicial Sets, Coverings}

\begin{abstract}

\end{abstract}

\maketitle

%%%%%%%%%%%%%%%%%%%%%%%%%%%%%%%%%%%%%%%%%%%%%%%%%%%%%%%%%%%%%%%%%%%%
\section*{Introduction}



%%%%%%%%%%%%%%%%%%%%%%%%%%%%%%%%%%%%%%%%%%%%%%%%%%%%%%%%%%%%%%%%%%%%
\section{Preliminaries}

We denote by $\mathcal{S}_f$ the category of finite sets.

We denote by $\Delta$ the simplex category and by $s\mathcal{S}$ the category of simplicial sets.

\cite{goerss2009simplicial}
\cite{may1992simplicial}

An abstract simplicial complex is a pair $(S,\mathcal{K})$ where $S$ is a set and $\mathcal{K}$ is a family of non-empty finite subsets of $S$ such that, if $\sigma\subseteq \tau$ and $\tau\in \mathcal{K}$ then $\sigma\in\mathcal{K}$. A morphism between of abstract simplicial complex $(S_1,\mathcal{K}_1)$ and $(S_2,\mathcal{K}_2)$

We shall recall that, given a topological space $X$, a covering space on $X$ it's a continuous map $p\colon E\to X$, such that for every $x\in X$, there is an open neighborhood $U$ such that $p^{-1}(U)$ it's a disjoint union of open sets $U_{\lambda}$, $\lambda\in\Lambda$, and $p|_{U_{\lambda}}\colon U_{\lambda}\to U$ it's a homeomorphism.

%%%%%%%%%%%%%%%%%%%%%%%%%%%%%%%%%%%%%%%%%%%%%%%%%%%%%%%%%%%%%%%%%%%%
\section{Abstract Simplicial Coverings}

%%%%%%%%%%%%%%%%%%%%%%%%%%%%%%%%%%%%%%%%%%%%%%%%%%%%%%%%%%%%%%%%%%%%
\section{Simplicial Coverings}

\begin{definition}
Let $X$ be a simplicial set. A simplicial covering of $X$ is a pair $(Y,p)$ where $Y$ is a simplicial set and $p\colon Y\longrightarrow X$ is a simplicial map such that...
\end{definition}

\begin{proposition}
Let $X$ be a simplicial set and $(Y,p)$ a simplicial covering of $X$. Then $(\vert Y\vert,\vert p\vert)$ is covering of $\vert X\vert$.
\end{proposition}

\begin{proof}

\end{proof}


%%%%%%%%%%%%%%%%%%%%%%%%%%%%%%%%%%%%%%%%%%%%%%%%%%%%%%%%%%
\bibliography{biblio}
\bibliographystyle{plain}



\end{document}

%------------------------------------------------------------------------------
% End of journal.tex
%------------------------------------------------------------------------------
