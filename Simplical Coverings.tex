\documentclass{amsart}

\usepackage{lipsum}
\usepackage{amsfonts}
\usepackage{graphicx}
\usepackage{amsmath}
\usepackage{amssymb}
\usepackage[english]{babel}
\usepackage{tikz-cd}
\usepackage{hyperref}
\usepackage{enumerate}


\newtheorem{theorem}{Theorem}[section]
\newtheorem{lemma}{Lemma}[section]
\newtheorem{proposition}{Proposition}[section]
\newtheorem{corollary}{Corollary}[section]
\newtheorem{definition}{Definition}[section]
\newtheorem{observation}{Observation}[section]
\newtheorem{example}{Example}[section]

\newenvironment{AMS}{****}{****}

\begin{document}

\title{Simplicial Coverings}

\author{Frank Murphy-Hernandez}
\address{Facultad de Ciencias, UNAM, Mexico City}
\email{murphy@ciencias.unam.mx}

\subjclass[2000]{Primary ****, ****; Secondary ****, ****}

\date{\today}

\keywords{Simplicial Sets, Coverings}

\begin{abstract}

\end{abstract}

\maketitle

%%%%%%%%%%%%%%%%%%%%%%%%%%%%%%%%%%%%%%%%%%%%%%%%%%%%%%%%%%%%%%%%%%%%
\section*{Introduction}



%%%%%%%%%%%%%%%%%%%%%%%%%%%%%%%%%%%%%%%%%%%%%%%%%%%%%%%%%%%%%%%%%%%%
\section{Preliminaries}




We recall that, given a topological space $X$, a covering space on $X$ it's a continuous map $p\colon E\to X$, such that for every $x\in X$, there is an open neighborhood $U$ such that $p^{-1}(U)$ it's a disjoint union of open sets $U_{\lambda}$, $\lambda\in\Lambda$, and $p|_{U_{\lambda}}\colon U_{\lambda}\to U$ it's a homeomorphism. We recommend the book of P. May \cite{may1999concise} as reference of covering spaces.


An abstract simplicial complex is a pair $(S,\mathcal{K})$ where $S$ is a set and $\mathcal{K}$ is a family of non-empty finite subsets of $S$ such that, if $\sigma\subseteq \tau$ and $\tau\in \mathcal{K}$ then $\sigma\in\mathcal{K}$. If $\sigma\in \mathcal{K}$, then the dimension of $\sigma$ is $\vert\sigma\vert-1$, and we denote it by $dim(\sigma)$. The elements of $\mathcal{K}$ of dimension $n$ are called $n$-simplices, and we denote the set of $n$-simplices by $\mathcal{K}_n$. The $0$-simplices are called vertices.

The geometric realization of an abstract simplicial  complex $(S,\mathcal{K})$ is the set of all function $\phi\colon S\longrightarrow [0,1]$ such that:
\begin{itemize}
\item $supp(\phi)\in\mathcal{K}$
\item $\sum_{s\in S}\phi(s)=1$
\end{itemize}
We denote this set by $\vert (S,\mathcal{K})\vert$. We may think $[0,1]^S$ as the direct limit of $[0,1]^A$ where $A$ranges over all finite subsets of $S$. So we give the $\vert (S,\mathcal{K})\vert$ the subspace topology.

 A morphism between  abstract simplicial complexes $(S_1,\mathcal{K}_1)$ and $(S_2,\mathcal{K}_2)$ is a map $f\colon S_1\longrightarrow S_2$ such that $f(\sigma)\in \mathcal{K}_2$ for any $\sigma\in\mathcal{K}_2$.



%%%%%%%%%%%%%%%%%%%%%%%%%%%%%%%%%%%%%%%%%%%%%%%%%%%%%%%%%%%%%%%%%%%%
\section{Abstract Simplicial Coverings}


\begin{definition}
Let $(S,\mathcal{K})$ be an abstract simplicial complex. An abstract simplicial covering of $(S,\mathcal{K})$ is a pair $((T,\mathcal{L}),p)$ where $(T,\mathcal{L})$ is a abstract simplicial complex and $p\colon T\longrightarrow S$ is a morphism of abstract simplicial complexes such that...
\end{definition}

\begin{proposition}
Let $(S,\mathcal{K})$ be an abstract simplicial complex and $((T,\mathcal{L}),p)$ an abstract simplicial covering of $(S,\mathcal{K})$. Then $(\vert (T,\mathcal{L})\vert,\vert p\vert)$ is covering of $\vert (S,\mathcal{K})\vert$.
\end{proposition}

\begin{proof}

\end{proof}



%%%%%%%%%%%%%%%%%%%%%%%%%%%%%%%%%%%%%%%%%%%%%%%%%%%%%%%%%%
\bibliography{biblio}
\bibliographystyle{plain}



\end{document}

%------------------------------------------------------------------------------
% End of journal.tex
%------------------------------------------------------------------------------
