\documentclass{amsart}

\usepackage{lipsum}
\usepackage{amsfonts}
\usepackage{graphicx}
\usepackage{amsmath}
\usepackage{amssymb}
\usepackage[english]{babel}
\usepackage{tikz-cd}
\usepackage{hyperref}
\usepackage{enumerate}


\newtheorem{theorem}{Theorem}[section]
\newtheorem{lemma}{Lemma}[section]
\newtheorem{proposition}{Proposition}[section]
\newtheorem{corollary}{Corollary}[section]
\newtheorem{definition}{Definition}[section]
\newtheorem{observation}{Observation}[section]
\newtheorem{example}{Example}[section]

\newenvironment{AMS}{****}{****}

\begin{document}

\title{Simplicial Coverings}

\author{Frank Murphy-Hernandez\\Luis Alberto Mac\'ias-Barrales}
\address{Facultad de Ciencias, UNAM, Mexico City}
\email{murphy@ciencias.unam.mx}

\subjclass[2000]{Primary ****, ****; Secondary ****, ****}

\date{\today}

\keywords{Simplicial Sets, Coverings}

\begin{abstract}

\end{abstract}

\maketitle

%%%%%%%%%%%%%%%%%%%%%%%%%%%%%%%%%%%%%%%%%%%%%%%%%%%%%%%%%%%%%%%%%%%%
\section*{Introduction}



%%%%%%%%%%%%%%%%%%%%%%%%%%%%%%%%%%%%%%%%%%%%%%%%%%%%%%%%%%%%%%%%%%%%
\section{Preliminaries}

We denote by $\mathcal{S}_f$ the category of finite sets.

We denote by $\Delta$ the simplex category and by $s\mathcal{S}$ the category of simplicial sets.

\cite{goerss2009simplicial}
\cite{may1992simplicial}

An abstract simplicial complex is a pair $(S,\mathcal{K})$ where $S$ is a set and $\mathcal{K}$ is a family of non-empty finite subsets of $S$ such that, if $\sigma\subseteq \tau$, $\sigma\neq\emptyset$ and $\tau\in \mathcal{K}$ then $\sigma\in\mathcal{K}$. A morphism between abstract simplicial complexes $f\colon(S_1,\mathcal{K}_1)\to(S_2,\mathcal{K}_2)$ it's a funtion $f\colon S_{1}\to S_{2}$ such that $f(\sigma)\in\mathcal{K}_{2}$ for any simplex $\sigma\in\mathcal{K}_{1}$. We will denote $\mathcal{C}$ to the category of abstract simplicial complexes.

The geometric realization of an abstract simplical complex $(S,\mathcal{K})$ is given by the following formula: first we give a total order to $S$. Then, for any simplex $\sigma=\{s_{0}<s_{1}<\ldots<s_{q}\}$ we define $|\sigma|=\Delta^{q}$, the standar topological $q$-simplex and we asociate to the vertex $s_{q}$ the $q$-th vertex of $\Delta^{q}$. If $\tau=\{s_{0}<\ldots<s_{q}\}$ is a simplex and $\sigma=\{s_{q_{1}}<\ldots<s_{q_{k}}\}\subseteq\tau$, we define $i_{\sigma}^{\tau}\colon|\sigma|\to|\tau|$ to be the affine function such that maps the $j$-th vertex of $|\sigma|$ to the $q_{j}$-th vertex of $|\tau|$. Thus we take $|S|$ as the colimit over this system. If $f\colon(S_{1},\mathcal{K}_{1})\to(S_{2},\mathcal{K}_{2})$ it's a morphism of abstract simplicial complexes, then we can define $|f|\colon|S_{1}|\to|S_{2}|$ as the colimit of the affine functions $|f|_{\sigma}|\colon|\sigma|\to|f(\sigma)|$. $|-|\colon\mathcal{C}\to\mathcal{T}$ it's a functor, were $\mathcal{T}$ is the category of topological spaces and continuous maps.


We shall recall that, given a topological space $X$, a covering space on $X$ it's a continuous map $p\colon E\to X$, such that for every $x\in X$, there is an open neighborhood $U$ such that $p^{-1}(U)$ it's a disjoint union of open sets $U_{\lambda}$, $\lambda\in\Lambda$, and $p|_{U_{\lambda}}\colon U_{\lambda}\to U$ it's a homeomorphism.

%%%%%%%%%%%%%%%%%%%%%%%%%%%%%%%%%%%%%%%%%%%%%%%%%%%%%%%%%%%%%%%%%%%%
\section{Abstract Simplicial Coverings}

\begin{definition}
Let $(S,\mathcal{K})$ an abstract simplicial complex. An abstract simplicial covering on $(S,\mathcal{K})$ it's a simplicial map $p\colon(E,\mathcal{L})\to(S,\mathcal{K})$ such that for each simplex $\sigma\in\mathcal{K}$, $p^{-1}(\sigma)=\amalg_{\lambda\in\Lambda}\sigma_{\lambda}$ were $\sigma_{\lambda}\in\mathcal{L}$ and $p|_{\sigma_{\lambda}}\colon \sigma_{\lambda}\to \sigma$ it is onto and one to one, for each $\lambda\in\Lambda$.
\end{definition}

\begin{proposition}
Let $p\colon(E,\mathcal{L})\to(S,\mathcal{K})$ be an abstract simplicial covering. Then $|p|:|E|\to|S|$ it's a covering space.
\end{proposition}

%%%%%%%%%%%%%%%%%%%%%%%%%%%%%%%%%%%%%%%%%%%%%%%%%%%%%%%%%%%%%%%%%%%%
\section{Simplicial Coverings}

\begin{definition}
Let $X$ be a simplicial set. A simplicial covering on $X$ is a pair $(Y,p)$ where $Y$ is a simplicial set and $p\colon Y\longrightarrow X$ is a simplicial map such that 
\end{definition}

\begin{proposition}
Let $X$ be a simplicial set and $(Y,p)$ a simplicial covering of $X$. Then $(\vert Y\vert,\vert p\vert)$ is covering of $\vert X\vert$.
\end{proposition}

\begin{proof}

\end{proof}


%%%%%%%%%%%%%%%%%%%%%%%%%%%%%%%%%%%%%%%%%%%%%%%%%%%%%%%%%%
\bibliography{biblio}
\bibliographystyle{plain}



\end{document}

%------------------------------------------------------------------------------
% End of journal.tex
%------------------------------------------------------------------------------
