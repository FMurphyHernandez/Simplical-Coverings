\documentclass{amsart}

\usepackage{lipsum}
\usepackage{amsfonts}
\usepackage{graphicx}
\usepackage{amsmath}
\usepackage{amssymb}
\usepackage[english]{babel}
\usepackage{tikz-cd}
\usepackage{hyperref}
\usepackage{enumerate}


\newtheorem{theorem}{Theorem}[section]
\newtheorem{lemma}{Lemma}[section]
\newtheorem{proposition}{Proposition}[section]
\newtheorem{corollary}{Corollary}[section]
\newtheorem{definition}{Definition}[section]
\newtheorem{observation}{Observation}[section]
\newtheorem{example}{Example}[section]

\newenvironment{AMS}{****}{****}

\begin{document}

\title{Geometric Realization of Covering Complexes}

\author{Frank Murphy-Hernandez}
\address{Facultad de Ciencias, UNAM, Mexico City}
\email{murphy@ciencias.unam.mx}

\author{Alberto Macias}
\address{Instituto de Matem\'aticas, UNAM, Mexico City}
\email{}

\subjclass[2000]{Primary ****, ****; Secondary ****, ****}

\date{\today}

\keywords{Abstract Simplicial Complexes, Coverings, Fundamental Group}

\begin{abstract}
We prove that the geometric realization of a covering complex is a covering space. Also, this holds true for the universal covering complex. Finally, we prove that the fundamental group of a complex coincides with the fundamental group of the geometric realization.
\end{abstract}

\maketitle

%%%%%%%%%%%%%%%%%%%%%%%%%%%%%%%%%%%%%%%%%%%%%%%%%%%%%%%%%%%%%%%%%%%%
\section*{Introduction}



%%%%%%%%%%%%%%%%%%%%%%%%%%%%%%%%%%%%%%%%%%%%%%%%%%%%%%%%%%%%%%%%%%%%
\section{Preliminaries}




We recall that, given a topological space $X$, a covering space on $X$ it's a continuous map $p\colon E\to X$, such that for every $x\in X$, there is an open neighborhood $U$ such that $p^{-1}(U)$ it's a disjoint union of open sets $U_{\lambda}$, $\lambda\in\Lambda$, and $p|_{U_{\lambda}}\colon U_{\lambda}\to U$ it's a homeomorphism. We recommend the book of P. May \cite{may1999concise} and the book of J. Rotman \cite{rotman2013introduction} as reference of covering spaces.


An abstract simplicial complex is a pair $(S,\mathcal{K})$ where $S$ is a set and $\mathcal{K}$ is a family of non-empty finite subsets of $S$ such that, if $\sigma\subseteq \tau$ and $\tau\in \mathcal{K}$ then $\sigma\in\mathcal{K}$. We call complexes to the abstract simplicial complexes.  If $\sigma\in \mathcal{K}$, then the dimension of $\sigma$ is $\vert\sigma\vert-1$, and we denote it by $dim(\sigma)$. The elements of $\mathcal{K}$ of dimension $n$ are called $n$-simplices, and we denote the set of $n$-simplices by $\mathcal{K}_n$. The $0$-simplices are called vertices. An edge $e$ in $(S,\mathcal{K})$ is a pair of vertices  $(x,y)$ where $\{x,y\}\in\mathcal{K}$, $x$ is the origin of the edge $e$ and we denote it by $orig(e)$, and $y$ is the end of the edge $e$ and we denote by $end(e)$. A path $\alpha$ in $(S,\mathcal{K})$ is a finite sequence of edges $e_1,\dots,e_n$ such that $end(e_i)=orig(e_{i+1})$ with $i=1,\dots,n-1$. We define $orig(\alpha)=orig(e_1)$ and $end(\alpha)=end(e_n)$. 

The geometric realization of a complex $(S,\mathcal{K})$ is the set of all function $\phi\colon S\longrightarrow [0,1]$ such that:
\begin{itemize}
\item $supp(\phi)\in\mathcal{K}$
\item $\sum_{s\in S}\phi(s)=1$
\end{itemize}
We denote this set by $\vert (S,\mathcal{K})\vert$. We may think $[0,1]^S$ as the direct limit of $[0,1]^A$ where $A$ranges over all finite subsets of $S$. So we give the $\vert (S,\mathcal{K})\vert$ the subspace topology.

 A morphism between complex $(S_1,\mathcal{K}_1)$ and $(S_2,\mathcal{K}_2)$ is a map $f\colon S_1\longrightarrow S_2$ such that $f(\sigma)\in \mathcal{K}_2$ for any $\sigma\in\mathcal{K}_2$.



%%%%%%%%%%%%%%%%%%%%%%%%%%%%%%%%%%%%%%%%%%%%%%%%%%%%%%%%%%%%%%%%%%%%
\section{Geometric Realization of Covering Complexes}

\begin{definition}
Let $(S,\mathcal{K})$ be a complex. We say that $(S,\mathcal{K})$ is connected if for any pair of vertices $x,y$ of $(S,\mathcal{K})$ there is a path $\alpha$ such that $orig(\alpha)=x$ and $end(\alpha)=y$.
\end{definition}


The following definition is due J. Rotman in \cite{rotman1973covering}.

\begin{definition}
Let $(S,\mathcal{K})$ be a complex. A covering of $(S,\mathcal{K})$ is a pair $((T,\mathcal{L}),p)$ where $(T,\mathcal{L})$ is a complex and $p\colon T\longrightarrow S$ is a morphism of complexes such that:
\begin{itemize}
\item $(T,\mathcal{L})$ is a connected complex.
\item For every $\sigma\in\mathcal{L}$, $p^{-1}(\sigma)=\bigcup_{i\in I}\sigma_i$ where the family $\{\sigma_i\}_{i\in I}$ is a pairwise disjoint family of simplices of $\mathcal{K}$ such that $p|_{\sigma_i}\colon \sigma_i\longrightarrow \sigma$ is bijective.
\end{itemize}
The map $p$ is called projection and the simplices $\sigma_i$ are called sheets over $\sigma$.
\end{definition}

\begin{proposition}
Let $(S,\mathcal{K})$ be an abstract simplicial complex and $((T,\mathcal{L}),p)$ an abstract simplicial covering of $(S,\mathcal{K})$. Then $(\vert (T,\mathcal{L})\vert,\vert p\vert)$ is covering of $\vert (S,\mathcal{K})\vert$.
\end{proposition}

\begin{proof}

\end{proof}

%%%%%%%%%%%%%%%%%%%%%%%%%%%%%%%%%%%%%%%%%%%%%%%%%%%%%%%%%%%%%%%%%%%%
\section{Geometric Realization of the Universal Covering Complex}

%%%%%%%%%%%%%%%%%%%%%%%%%%%%%%%%%%%%%%%%%%%%%%%%%%%%%%%%%%%%%%%%%%%%
\section{Fundamental Group}

%%%%%%%%%%%%%%%%%%%%%%%%%%%%%%%%%%%%%%%%%%%%%%%%%%%%%%%%%%
\bibliography{biblio}
\bibliographystyle{plain}



\end{document}

%------------------------------------------------------------------------------
% End of journal.tex
%------------------------------------------------------------------------------
